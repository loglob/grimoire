% TeX interface to the Grimoire material parser
% On TeX side these commands don't do anything.
% On Grimoire side, when used in a table, they feed necessary metadata to the material parser.
%
% When a unit is required, an abbreviation macro (i.e. \gr) or literal unit string (i.e. [g]) is allowed.
% You can also omit the unit entirely with defaults to the dimensionless unit (1).
%
% A material line may contain exactly one of:
% - one \grUnit and up to one \grOtherUnit
% - one \grVariants

% Declares a name for the material defined by this table row.
% Usually optional, only required when one row defines multiple materials, or to overwrite the typeset name.
% Does not typeset any content.
% #1 - The material name
\newcommand{\grMaterial}[1]{}

% Declares the "packet size" of a material.
% Defaults to dimensionless 1 if not given.
% Typesets its argument unmodified unless * is given.
% #1 - (star) if given, don't typeset anything
% #2 - A number followed by a unit, or just a number
\NewDocumentCommand{\grUnit}{s m}{
	\IfBooleanF{#1}{#2}
}

% Declares a secondary unit. Used together with \grMainUnit.
% Used like: " A pack of \grOtherUnit{12 flasks} with \grUnit{20 \ml} each " (order irrelevant)
% Typesets its argument unmodified unless * is given.
% #1 - (star) if given, don't typeset anything
% #2 - A number followed by a name, or just a number
\NewDocumentCommand{\grOtherUnit}{s m}{
	\IfBooleanF{#1}{#2}
}

% Sets up shorthanded /-separated variant lists
% Used like " \grDeclareVariants{small/medium/large} \\ Rocks & \grVariants{ 1 \Cu/10 \Cu/2 \Ag } "
% Typesets its argument unmodified
% #1 - Names separated by '/'
\newcommand{\grDeclareVariants}[1]{#1}

% Applies the last \grDeclareVariants
% Typesets its argument unmodified
% #1 - Prices separated by the separator specifie din \grDeclareVariants
\newcommand{\grVariants}[1]{#1}

% Post-processes already emitted materials by applying conversion rules.
% Doesn't typeset anything.
% BOTH units may be blank to perform unit-preserving renaming.
% #1 - A wildcard for the base material. Must contain exactly one '*'.
% #2 - A number and unit specifying how much of the input material is required.
% #3 - A pattern for the resulting material. Must contain exactly one '*' which is substituted with the value of * in #1.
% #4 - A number and unit specifying how much of the output material is produced.
\newcommand{\grPost}[4]{}
